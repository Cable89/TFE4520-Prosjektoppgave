\section{Assembly}

\subsection{Printed circuit board production}
Modern printed circuit board (PCB) production is a complicated process involving many different chemicals \citep{pcbprod}. The innermost part of the PCB is made of a fiberglass compound that is plated with copper. This copper layer is further coated with a layer of photoreactive film. This film is then exposed to light at the correct places so that when the glass fiber copper laminates are exposed to acid, the copper is removed in the places desired. Then holes are drilled with a CNC before they are electrolyse plated with copper. Additional copper layers can be added with the same procedure at this point. After all copper layers are in place, the board is coated with a solder mask, which is a lacquer resistant to solder, so that when soldering the card the solder does not wander to the wrong places. Then a silkscreen printer prints text, designators and comments. At least, the components are finally soldered to the board, either manually or with CNC machines. This process (with or without soldering) is usually done by a third party.