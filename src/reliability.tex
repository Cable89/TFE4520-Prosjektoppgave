\subsection{Reliability}
\todo{Sources}
\todo{Lifecycle management}
\todo{Reliability model, fault tree diagram}
In every system there is certain demands for reliability (reliablilty describes the ability of a system to function under certain condiditons for a specified period of time). The DRSSTC should function in indoor conditions and dry outdoors weather, the DRSSTC should also function after sitting in storage for up to one year. And should function after transportation by trolley and or car. And after sitting on a stage during rigging. The DRSSTC should function for several years with transportation and rigging up to 30 times a year.

To achieve this the system should also have high maintainability. It should take a maximum of one day(downtime?) to fix an error (even errors provoked by design errors).

By design errors we mean errors that still allow operation for some time before failing.

By using a back plate design and keeping backup modules the downtime demand should be met. \todo{Mer analytisk}