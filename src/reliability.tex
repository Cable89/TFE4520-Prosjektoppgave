\subsection{Reliability}
\label{reliability}
In every system there is certain demands for reliability (reliablilty describes the ability of a system to function under certain condiditons for a specified period of time). The DRSSTC should function in indoor conditions and dry outdoors weather, the DRSSTC should also function after sitting in storage for up to one year. And should function after transportation by trolley and or car. And after sitting on a stage during rigging. The DRSSTC should function for several years with transportation and rigging up to 30 times a year.

To achieve this the system should also have high maintainability. It should take a maximum of one day to fix an error.

\subsubsection{Fault tree}
A method for analyzing reliability of a system is using a fault tree \citep{faulttree}. The fault trees for this system is shown in \cref{fig:ftree_drsstc_dest} and \cref{fig:ftree_drsstc}. The inputs are failure sources that can occur these are listed in \cref{tab:fail_sources}. The probability P for each failure source is calculated from the number of times this failure has occurred divided by the number times the system has been used. These numbers are extracted from the log entries in Omega Verksteds log book for their tesla coil \citep{teslalog}. A total of 33 uses are logged in the period 2013-08-22 to 2016-11-07, the real number of uses should be assumed higher as the log book is not written at each use due to human error. The fault sources that does not occur in the log book is assumed to have a probability P of less than 0,03. Some of these failure sources are due to operator error and is outside of this report to fix. These failures sources are F3, F5, and F7 they are attempted to fix with procedures \citep{teslaprosedyrer}. The failure source with the highest probability $P=0,12$ is F8 this is a failure that does not occur by itself but as a consequence of other failure sources as shown in \cref{fig:ftree_drsstc_dest} and must thus be fixed by addressing the failure sources leading to this failure. The failure source remaining with the highest probability $P=0,09$ is F2 defective optical cable. By using the standard rules of combining probabilities with logic functions we get a calculated probabiliy of failure of P(F)= 0,71. The probability for destructive failure P(F8) used in calculating the probability for failure P(F) is the value gotten from calculating the output from the destructive failure fault tree P(F8)=0,666, and not the value listed in \cref{tab:fail_sources}.

\begin{table}[h]
    \centering
    \begin{tabular}{|p{0.05\textwidth}|p{0.1\textwidth}|p{0.75\textwidth}|}
        \hline
        & P & Description \\ \hline
        F0 & 0,03 & System is transported and vibrations shakes loose the wires on the high voltage side of the system.  \\ \hline
        F1 & <0,03 & No carrier wave (CW) detection is implemented in the optical reciever of the driver TK500, or this detection is defective.\\ \hline
        F2  & 0,09 & Optical cable is defective (someone has stepped on it).\\ \hline
        F3  & 0,03 & Optical cable is missing.\\ \hline
        F4  & < 0,03 & Pulse shaper (TK100) is defective.\\ \hline
        F5  & 0,06 & Pulse shaper is set to too high volume.\\ \hline
        F6  & 0,03 & Power limiter is defective.\\ \hline
        F7  & 0,06 & Power limiter is set to too high power.\\ \hline
        F8  & 0,12 & Destructive failure happens, power amplifier or the load capacitor C1 catches fire. \\ \hline
        F9  & 0,03 & Internal cables are disconnected due to vibrations while transporting the system. \\ \hline
    \end{tabular}
    \caption{Failure sources}
    \label{tab:fail_sources}
\end{table}

\begin{figure}[h]
\begin{circuitikz}[scale = 0.65, transform shape] \draw
(0,0) node[] (x0){F7: Limiter settings}
(0,1.2) node[] (x1){F6: Limiter def.}
(0,2) node[] (x2){F5: Sig. gen settings}
(0,2.8) node[] (x3){F4: Sig. gen def.}
(0,4) node[] (x4){F3: Optical cable missing}
(0,5) node[] (x5){F2: Optical cable def.}
(0,6) node[] (x6){F1: CW detection def.}
(0,7) node[] (x7){F0: HV cables disconnected}
(15,2) node[] (x8){F8: Destructive failure}
(4,6.5) node[or port] (myor1) {}
(4,4.5) node[or port] (myor2) {}
(4,2.5) node[or port] (myor3) {}
(6,1.5) node[and port] (myand1) {}
(8,2.5) node[or port] (myor4) {}
(10,3) node[and port] (myand2) {}
(12,2) node[or port] (myor5) {}
(x0) -- (10,0) -- (myor5.in 2)
(x1) -- (myand1.in 2)
(x2) -- (myor3.in 2)
(x3) -- (myor3.in 1)
(x4) -- (myor2.in 2)
(x5) -- (myor2.in 1)
(x6) -- (myor1.in 2)
(x7) -- (myor1.in 1)
(myor5.out) -- (x8)
(myor1.out) -- (myand2.in 1)
(myor2.out) -- (myor4.in 1)
(myor3.out) -- (myand1.in 1)
(myand1.out) -- (myor4.in 2)
(myor4.out) -- (myand2.in 2)
(myand2.out) -- (myor5.in 1);
\end{circuitikz}
\caption{Fault tree for destructive failure of DRSSTC}
    \label{fig:ftree_drsstc_dest}
\end{figure}

\begin{figure}[h]
\begin{circuitikz} \draw
(0,2) node[] (x0){F8: Destructive failure}
(0,1) node[] (x1){F9: Internal cables missing}
(0,0) node[] (x2){F2: Optical cable defective}
(8,1.5) node[] (x3){Failure}
(4,0.5) node[or port] (myor1) {}
(6,1.5) node[or port] (myor2) {}
(x0) -- (myor2.in 1)
(x1) -- (myor1.in 1)
(x2) -- (myor1.in 2)
(myor1.out) -- (myor2.in 2)
(myor2.out) -- (x3);
\end{circuitikz}
\caption{Fault tree for non-destructive failure of DRSSTC}
    \label{fig:ftree_drsstc}
\end{figure}

% Fault tree diagram

% Utgangstrinn blåser = For mye effekt = (limiter feil innstilt || limiter defekt ) && ( (Signalgenerator feil innstilt || sig. gen defekt) || (optisk kabel detti ut && bærebølgedeteksjon def.) )
% ledninger mellom moduler detter ut
% Mosfetdrivere blåser = ?
% utgangskondensator blåser = For lav rating på kondis ? for høy effekt
% optisk inn dør = tråkket på
% overslag mellom ting på hv siden etter risting
% utgangskabelen smelter

\subsubsection{Maintainability}
The goal of errors taking a maximum of one day to fix is set so that you will always have a functioning system by testing it one day before it is to be used for a performance. By partitioning the system into modules with limited components, the assembly time for one module is kept short. By keeping spare modules, and spare components. You will be able to either just switch the module, wich by the use of a backplane takes only a matter of minutes. Or assemble a new module wich should take less than one day.