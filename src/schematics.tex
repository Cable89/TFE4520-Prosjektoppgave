\section{Schematics}
The schematics for this system is drawn in Altium designer and is attached in appendix B. Note that appendix B is best printed on A3 paper. All digital produced material have been stored in a central Git-based version control system which is accessible at \citep{githubtesla}. The schematics is found under '05\_Hardware\/TK500\_Driver\/ProjectOutputs\/TK500\_design\_documentation.PDF' Also note that not all of the schematics are finished, and that the schematics are in norwegian.

\subsection{Altium designer}
\label{altium}
Altium Designer \citep{altium} is a powerful CAD tool for designing printed circuit boards (PCB). This type of tool allows the designer to first design or import each component one by one into the program, that also is used in the full design. When this is finished, the the schematics can be drawn and synced with a layout tool.

Alternatives to Altium Designer is Eagle \citep{eagle}, kicad \citep{kicad}, Proteus \citep{proteus}, ultiboard \citep{ultiboard} and orcad \citep{orcad}. The reasons for Altium designer to be chosen over the others is Eagle, and kicad (where  kicad is considered an open source version of Eagle) is that Eagle has very few automations, thus you end up spending a lot more time to do the same work. The rest of the programs were not chosen of more subjective reasons. In addition to these reasons these CAD programs are very expensive so a key point in choosing Altium Designer is that NTNU has a licence for students. Also we have previous experience using Altium Designer, both from hobby work and relevant summer internships.