\section{Specification}
The system should take audio as input and output acoustic audio by means of an high voltage electric discharge in air creating plasma by the means of a resonant transformer (tesla coil). The system should also be designed such that it will always be functional, and in the case of the system not being functional should fail in a safe and non-destructive way (safe having priority over non-destructive). As well as indicating witch part of the system is non-functional. The system should also be designed in such a way that this non-functioning part may be swapped for a spare functioning part.

With this specification in mind the system has been partitioned into sub-systems listed below. The sub systems has been assigned an arbitrary part number starting in "TK".

\begin{list}
    \item TK100 Signalgenerator
    \item TK500 Driver
    \begin{list}
        \item TK501 Frontpanel
        \item TK502 Frontpanel LEDS
        \item TK510 Signalbakplan
        \item TK511 Blindkort
        \item TK512 Optisk Mottaker
        \item TK513 Limiter
        \item TK514 Interrupter
        \item TK516 Ekstra-PSU
        \item TK517 P5V0-PSU
        \item TK518 P18V-PSU
        \item TK519 Spenningsvakt
        \item TK520 GDT-Trafo
        \item TK530 Kraftbakplan
        \item TK531 Utgangstrinn
        \item TK532 Kondensatorkort
    \end{list}
    \item TK1000 Spolerigg
\end{list}

\subsection{TK100 Signalgenerator}
Should take a wide range of inputs, and output the triggering signal for the driver. Modulated on a carrier wave.
\subsubsection{Triggering signal}
The triggering signal should be in the hearable frequency range with a maximal pulse width (high) of $680\mu S$.
\subsubsection{Carrier wave}
The carrier wave should have a sufficently high frequency as according to the nyquist-shannon sampling theorem, in addition to be easily detected in the recieving end.
\subsubsection{Modulation}
The triggering signal should be pulse width modulated with a logical high having a duty cycle of 80\% and logical low having a duty cycle of 20\%.

\subsection{TK500 Driver}

\subsubsection{TK501 Frontpanel}
Should have a height of 4 standard rack units (4U) (17.78cm). Partitioned into multiple parts.
\subsubsubsection{Signal backplane card matrix}
Use the entire height, height divided into equal parts for each card in the signal backplane card. Each division should contain four LEDs; Green (Card inserted), Red (Fault), Green (status), Yellow (led in button). A field for card name. Space for buttons depending on the needs of the specific card.

\subsubsubsection{Power backplane card matrix}
Three leds for each of the three cards in the power backplane. Green (card inserted), Red (fault), Green (status).

\subsubsubsection{Display}
Voltage meter for output voltage.
Ampere meter for output current.

\subsubsubsection{Power buttons}
Emergency stop
Spring loaded rotary switch: slow start
Momentary switch: Power on
Momentary switch: Power off

\subsection{TK510 Signalbakplan}
The siglan backplane should have six slots for signal modules, and three slots for power supplies.
\subsubsection{Signal module slot}
1 bus with straps for pull up and pull down on bus lines.
Card inserted detection
LED signals to front panel
Misc signals to front panel
Two contacts for signals other places inside chassis
Supply voltages and ground