\section{Specification}
The system has been partitioned into sub-systems listed below. The sub systems has been assigned an arbitrary part number with the prefix "TK".


\begin{enumerate}
    \item TK100 Signalgenerator
    \item TK500 Driver
    \begin{enumerate}
        \item TK501 Frontpanel
        \item TK502 Frontpanel LEDS
        \item TK510 Signalbakplan
        \item TK511 Blindkort
        \item TK512 Optisk Mottaker
        \item TK513 Limiter
        \item TK514 Interrupter
        \item TK516 Ekstra-PSU
        \item TK517 P5V0-PSU
        \item TK518 P18V-PSU
        \item TK519 Spenningsvakt
        \item TK520 GDT-Trafo
        \item TK530 Kraftbakplan
        \item TK531 Utgangstrinn
        \item TK532 Kondensatorkort
    \end{enumerate}
    \item TK1000 Spolerigg
\end{enumerate}

\subsection{TK100 Signal generator}
The Signal generator Should take a wide range of inputs and output the triggering signal X2 for the driver. It consists of the pulse shaper and signal source mentioned in \cref{DRSSTC} and shown in \cref{fig:func_block}. The output signal X2 should be sent through a optical plastic fibre to be against EMI as mentioned in \cref{sa} and \cref{optical}. The output signal should also be modulated on a carrier wave as described closer in \cref{triggering_signal} and \cref{carrier_wave}.

\subsection{TK500 Driver}

The driver should take the output of the signal generator (TK100) as input, and output a 160VDC signal at the resonant frequency of the coil rig while the input signal is high and the peak power output does not exceed a configurable level. The driver should also have the capacitor of the primary resonant circuit integrated.

\subsubsection*{TK501 Frontpanel}
Should have a height of 4 standard rack units (4U) (17.78cm). Partitioned into multiple parts.
\subsubsection*{Signal back plane card matrix}
Use the entire height, height divided into equal parts for each card in the signal back plane card. Each division should contain four LEDs; Green (Card inserted), Red (Fault), Green (status), Yellow (led in button). A field for card name. Space for buttons depending on the needs of the specific card.

\subsubsection*{Power back plane card matrix}
Three leds for each of the three cards in the power back plane. Green (card inserted), Red (fault), Green (status).

\subsubsection*{Display}
Voltage meter for output voltage.
Ampere meter for output current.

\subsubsection*{Power buttons}
Emergency stop
Spring loaded rotary switch: slow start
Momentary switch: Power on
Momentary switch: Power off

\subsection*{TK510 Signalbakplan}
The signal back plane should have six slots for signal modules, and three slots for power supplies.

\subsubsection*{Signal module slot}
The signal module slot should have the following signals:
\begin{itemize}
    \item 1 bus with straps for pull up and pull down on bus lines. (B1-B14)
    \item Card inserted detection (A1)
    \item LED signals to front panel (A1-A5)
    \item Misc signals to front panel (A6-A10)
    \item Two contacts for signals other places inside chassis (A11-A14)
    \item Supply voltages and ground (A15-A18 \& B15-B18)
\end{itemize}
The pinout on the module slot connector is shown in \cref{fig:signalmodulslot}

\begin{figure}[h]
    \centering
    \includegraphics[trim={4.1cm 14.3cm 21.2cm 3.3cm},clip,width=\textwidth]{img/TK510_Signalbakplan.pdf}
    \caption{Signal module card slot connector (detail from schematic of TK510 Signalbakplan)}
    \label{fig:signalmodulslot}
\end{figure}

\subsection*{TK511 Blindkort}
The purpose of $TK511$ is to take the space where no module is inserted into the signal back plane, and emulate an inserted module.

\subsection*{TK512 Optisk Mottaker}
The purpose of $TK512$ is to convert the optical signal input into the driver $TK500$ to an electrical signal, and a carrier wave detected signal.

\subsection*{TK513 Limiter}
The purpose of $TK513$ is to prevent the coil rig $TK1000$ from catching fire, and to keep the spark from turning yellow.

\subsection*{TK514 Interrupter}
The purpose of $514$ is to fire the tesla coil when receiving input signal and to tune the output signal via a positive feedback loop to the systems resonance frequency.

\subsection*{TK516 Ekstra-PSU}
The purpose of $TK516$ is to reserve a slot for future voltages.

\subsection*{TK517 P5V0-PSU}
The purpose of $TK516$ is to provide 5V DC to the driver $TK500$.

\subsection*{TK518 P18V-PSU}
The purpose of $TK516$ is to provide 18V DC to the driver $TK500$.

\subsection*{TK519 Spenningsvakt}
The purpose of $TK516$ is to monitor the supply voltages in the driver $TK500$.

\subsection*{TK520 GDT-Trafo}
The purpose of $TK516$ is to provide galvanic isolation between the signal back plane and the power back plane.

\subsection*{TK530 Kraftbakplan}
The purpose of $TK516$ is to provide the connections and monitoring for the output stage $TK531$ and the load capacitor $TK532$.

\subsection*{TK531 Utgangstrinn}
The purpose of $TK516$ is to step up the voltage on the output.

\subsection*{TK532 Kondensatorkort}
The purpose of $TK516$ is to provide the capacitance of the primary resonance circuit.

\subsection{TK1000 Spolerigg}
The purpose of $TK516$ is to look good and output the voltage arc or corona discharge.